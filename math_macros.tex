% derivatives
\newcommand{\dderiv}[2]{\frac{\mathrm{d}}{\mathrm{d}#2} (#1)}

% partial derivatives
\newcommand{\pderiv}[2]{\frac{\partial #1}{\partial #2}}
\newcommand*{\pd}[3][]{\ensuremath{\frac{\partial^{#1} #2}{\partial #3}}}

% Integral dx
\newcommand{\dx}{\mathrm{d}x}

% Evaluation operator
\newcommand*{\at}[2]{\Big|_{#1}^{#2}}

% Probability macros
% Variance
\newcommand{\Var}{\mathrm{Var}}
% Covariance
\newcommand{\Cov}{\mathrm{Cov}}
% Bias
\newcommand{\Bias}{\mathrm{Bias}}
% Probability
\newcommand*{\prob}[1]{\ensuremath{\mathbb{P}\left[#1\right]}}
\newcommand*{\expectation}[1]{\ensuremath{\mathbb{E}\left[#1\right]}}
\newcommand*{\variance}[1]{\ensuremath{\mathbb{V}\left[#1\right]}}
% Indicator function of an event
\newcommand*{\ind}[1]{\ensuremath{\mathds{I}_{\left[#1\right]}}}

% Optimization macros
\DeclareMathOperator{\dom}{Dom}
\DeclareMathOperator{\diag}{Diag}
\DeclareMathOperator{\prox}{prox}
\DeclareMathOperator*{\proj}{proj}
\DeclareMathOperator*{\sign}{sign}
\DeclareMathOperator*{\argmax}{argmax}
\DeclareMathOperator*{\argmin}{argmin}

\newcommand*{\pdef}{\ensuremath{\mathbb{S}_{++}}}
\newcommand*{\psdef}{\ensuremath{\mathbb{S}_{+}}}

%% \mathbb symbols
\DeclareMathOperator{\E}{\mathbb{E}}
\DeclareMathOperator{\A}{\mathbb{A}}
\DeclareMathOperator{\R}{\mathbb{R}}
\DeclareMathOperator{\C}{\mathbb{C}}
\DeclareMathOperator{\X}{\mathbb{X}}
\DeclareMathOperator{\N}{\mathbb{N}}
\DeclareMathOperator{\Q}{\mathbb{Q}}
\DeclareMathOperator{\V}{\mathbb{V}}

%% \mathcal symbols
\DeclareMathOperator{\RR}{\mathcal{R}}
\DeclareMathOperator{\PP}{\mathcal{P}}
\DeclareMathOperator{\NN}{\mathcal{N}}
\DeclareMathOperator{\CC}{\mathcal{C}}
\DeclareMathOperator{\FF}{\mathcal{F}}
\DeclareMathOperator{\YY}{\mathcal{Y}}
\DeclareMathOperator{\XX}{\mathcal{X}}
\DeclareMathOperator{\LL}{\mathcal{L}}

% limit arrow "goes to"
\newcommand{\goto}{\rightarrow}

% text above math symbol
\newcommand{\textabove}[2]{\mathrel{\overset{\makebox[0pt]{\mbox{\normalfont\tiny #1}}}{#2}}}

% Big O
\newcommand{\bigo}{{\mathcal O}}

% 1/2
\newcommand{\half}{\frac{1}{2}}
% 1/2
\newcommand{\halfof}[1]{\frac{#1}{2}}

% Real numbers
\newcommand\reals{{{\rm l} \kern -.15em {\rm R} }}

% Complex numbers
\newcommand\complex{{\raisebox{.043ex}{\rule{0.07em}{1.56ex}} \hskip -.35em {\rm C}}}

% Matrix functions
\DeclareMathOperator{\trace}{Tr}

% matrix environment for vectors or matrices where elements are centered
\newenvironment{mat}{\left[\begin{array}{ccccccccccccccc}}{\end{array}\right]}
\newcommand\bcm{\begin{mat}}
\newcommand\ecm{\end{mat}}

% matrix environment for vectors or matrices where elements are right justified
\newenvironment{rmat}{\left[\begin{array}{rrrrrrrrrrrrr}}{\end{array}\right]}
\newcommand\brm{\begin{rmat}}
\newcommand\erm{\end{rmat}}

% for left brace and a set of choices
\newenvironment{choices}{\left\{ \begin{array}{ll}}{\end{array}\right.}
\newcommand\when{&\text{if~}}
\newcommand\otherwise{&\text{otherwise}}
% sample usage:
%  \delta_{ij} = \begin{choices} 1 \when i=j, \\ 0 \otherwise \end{choices}

% For equations: first argument is a label, second is the equation 
\newcommand{\eql}[2]{\begin{equation}\label{eq:#1}\begin{split}#2\end{split}\end{equation}}
% Or just the equation 
\newcommand{\eq}[1]{\begin{equation}\begin{split}#1\end{split}\end{equation}}

% some useful macros for finite difference methods:
\newcommand\unp{U^{n+1}}
\newcommand\unm{U^{n-1}}
\newcommand\un{U^{n}}

% paired parenthesis
\DeclarePairedDelimiter\pa{(}{)}%
\DeclarePairedDelimiter\bra{[}{]}%
\DeclarePairedDelimiter\curly{\{}{\}}%
\DeclarePairedDelimiter\abs{\lvert}{\rvert}%
\DeclarePairedDelimiter\norm{\lVert}{\rVert}%
\DeclarePairedDelimiter\floor{\lfloor}{\rfloor}%
\DeclarePairedDelimiter\ceil{\lceil}{\rceil}%

% Theorems and co. environments setup
\newtheorem{definition}{Definition}
\newtheorem{theorem}{Theorem}
\newtheorem{corollary}[theorem]{Corollary}
\newtheorem{lemma}[theorem]{Lemma}


